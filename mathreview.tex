Annette Chau
Math Review

Section 1: Sinusoids 

1. Graph the function 0.5 sin(2π(2)t −π/2) by hand. (see attached)
2. Consider the difference between A sin(2πf t + φ) and −A sin(2πf t + φ).
(a) Explain how A sin(2πf t + φ) is transformed when the function is multiplied by −1.
	The graph is flipped over the x axis, or just phase shifted.
	
(b) Write an equivalent expression to −A sin(2πt + φ) that does not use any negative signs. Note the
frequency of 1. Hint: consider changing the phase!
	= A sin (-2πft - φ)
	= A sin (2πft - φ + π)

3. What is tan−1(−√3) if sin is negative?

(5π/3)

Section 2: Trig Identities
1. Show that csc(θ) cos(θ) tan(θ) = 1.
	= 1/sin(θ)cos(θ) tan (θ) = 1
	= cos (θ)/sin(θ) * sin(θ)/cos(θ) = 1
	= 1 = 1

2. Simplify (cot(x) cos(x)) / (tan(−x) sin(π/2 −x))
	= (cot(x)cos(x))/tan(-x)sin(π/2 - x)
	= (cot(x) cos(x))/tan(-x)cos(x)
	= cot(x)/-tan(x)
	= 1/tan^2(x)
	
3. Show that tan(x + y) = tan(x)+tan(y) / 1−tan(x) tan(y) starting from sin(x+y) / cos(x+y) and using sum and difference angle identities.
	= sin(x+y) / cos(x+y) = (sin(x)cos(y) + cos(x)sin(y) ) / (cos(x) cos(y) - sin(x) sin (y))
	This is the most confusing equation I have ever tried to type up. Essentially, divide both sides by cos(x) and cos (y)
	= (sin(x)cos(y)/cos(x) cos (y) + cos(x) sin (y)/cos(x)cos(y)) / ((cos (x) cos (y) / cos(x) cos(y)) - (sin(x) sin(y)/cos(x)cos(y)))
	= sin(x)/cos(x) + sin(y)/cos(y) / sin(x) sin(y)/cos(x) cos(y)
	= tan(x) + tan (y) / 1 - tan(x)tan(y)
	
Section 3: Summation Notation
1. 25
2. See attached
3. See attached

Section 4: Complex Numbers
1. Find the solutions to z^2 = −4.
	z = \sqrt{-4} = 2i
	
2. Consider x = 3 + 2i and y = 2 − i.
(a) What is x + y?
	= 3 + 2i + 2 - i
	= 5 - i
	
(b) What is xy?

	= (3+2i)(2-i)
	= 6 + 4i - 3i - 2i^2
	= i + 8